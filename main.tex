\documentclass[12pt,a4paper]{article}

% Encoding, fonts, language
\usepackage[utf8]{inputenc}
\usepackage[T1]{fontenc}
\usepackage[english]{babel}
\usepackage{lmodern}
\usepackage{microtype}

% Layout
\usepackage{geometry}
\geometry{margin=1in}
\setlength{\parskip}{0.6em}
\setlength{\parindent}{0pt}

% Tables and figures
\usepackage{booktabs}
\usepackage{tabularx}
\usepackage{graphicx}
\usepackage{caption}

% Lists and quotes
\usepackage{enumitem}
\usepackage{csquotes}

% Links and clever references
% Math first
\usepackage{amsmath}

% Links
\usepackage[hidelinks]{hyperref}
\usepackage{orcidlink}

% Clever references
\usepackage[capitalise,noabbrev]{cleveref}

% ---------------------------
% Bibliography (biblatex+biber)
% Default: numeric with superscript citations (numeric-super)
\usepackage[
  backend=biber,
  style=numeric-comp,
  citestyle=numeric-comp,
  sorting=nyt,
  giveninits=true,
  maxcitenames=2,
  maxbibnames=99,
  autocite=superscript
]{biblatex}
\addbibresource{references.bib}

% To switch to APA style:
% 1) Comment the biblatex line above and uncomment the two lines below.
% 2) Use \autocite (already used in the text) for inline APA parentheses.
% \usepackage[backend=biber,style=apa,sorting=nyt,giveninits=true,maxcitenames=2,autocite=inline]{biblatex}
% \DeclareLanguageMapping{english}{english-apa}

% ---------------------------
% Glossaries (professional glossary)
\usepackage[acronym]{glossaries-extra}
\setabbreviationstyle[acronym]{long-short}
\makeglossaries
\setglossarystyle{long}

% Acronyms
\newacronym{dht}{DHT}{Distributed Hash Table}
\newacronym{dlt}{DLT}{Distributed Ledger Technology}
\newacronym{fipaacl}{FIPA-ACL}{FIPA Agent Communication Language}
\newacronym{cnp}{CNP}{Contract Net Protocol}
\newacronym{mve}{MVE}{Minimum Viable Ecosystem}

% Glossary entries (main terms)
\newglossaryentry{student-agent}{
  name={Student Agent},
  description={Personal autonomous assistant optimizing a learner’s path toward verifiable mastery across a curriculum graph; initiates requests, negotiates resources, and tracks progress}
}
\newglossaryentry{curriculum-agent}{
  name={Curriculum Agent},
  description={Digital representation of a knowledge domain that exposes a dependency graph of concepts, curates resources, and generates mastery challenges}
}
\newglossaryentry{mentor-agent}{
  name={Mentor Agent},
  description={Provider of immediate, contextual explanations and guidance; may be an AI agent or a proxy for a human teacher (calendar, expertise, availability)}
}
\newglossaryentry{wellbeing-agent}{
  name={Well-being Agent},
  description={Agent that monitors and supports student well-being (time on task, frustration signals, breaks), escalating to human support when needed}
}
\newglossaryentry{pact}{
  name={Scholastic Pact on Temporal Non-Aggression},
  description={Ethical commitment to protect student time and redefine rigor as pursuit of verifiable mastery rather than time-on-task}
}
\newglossaryentry{symphony}{
  name={Symphony of Agents},
  description={Decentralized, intelligent ecosystem where autonomous agents orchestrate personalized learning aligned with the Pact}
}
\newglossaryentry{symphony-protocol}{
  name={Symphony Protocol},
  description={Beacon-based orchestration mechanism for dynamic, low-latency matching of educational needs with resources}
}
\newglossaryentry{beacon}{
  name={Beacon},
  description={Lightweight broadcast message advertising a requirement vector for local capability matching in a distributed network}
}
\newglossaryentry{mastery-credential}{
  name={Mastery Credential},
  description={Verifiable, cryptographically secured record on a ledger attesting to demonstrated competence for a specific objective or skill}
}
\newglossaryentry{permissioned-dlt}{
  name={Permissioned \glsentryshort{dlt}},
  description={Ledger where only authorized validators can record transactions; suitable for regulated educational contexts}
}
\newglossaryentry{bdi}{
  name={BDI (Belief–Desire–Intention)},
  description={Cognitive architecture enabling goal-oriented agents to act based on beliefs, commit to intentions, and pursue desires}
}

\title{A New Covenant for Learning: From Compulsion to Concert}

\author{Janusz Feszter\,\orcidlink{0009-0002-1330-7530}\\
\small \href{mailto:jfeshter@gmail.com}{jfeshter@gmail.com}\\
\small X: \href{https://x.com/YaffFesh}{@YaffFesh} \quad
\small GitHub: \href{https://github.com/jfesvd-crypto}{github.com/jfesvd-crypto}
}

\newcommand{\createddate}{13.09.2025}
\date{Created: \createddate\\Updated: \today}

% Math first
\usepackage{amsmath}

% Links
\usepackage[hidelinks]{hyperref}

% PDF metadata (tu wstaw)
\hypersetup{
  pdftitle={A New Covenant for Learning: From Compulsion to Concert},
  pdfauthor={Janusz Feszter (ORCID 0009-0002-1330-7530)}
  % opcjonalnie:
  % pdfsubject={Education, Multi-Agent Systems},
  % pdfkeywords={education, multi-agent, mastery learning, DLT}
}

% Clever references (musi być po hyperref)
\usepackage[capitalise,noabbrev]{cleveref}

\begin{document}
\maketitle
\tableofcontents
\newpage

\begin{table}[htbp]
\centering
\caption{Educational Paradigm Transformation}
\label{tab:paradigm_shift}
\small
\begin{tabularx}{\textwidth}{@{}lXX@{}}
\toprule
\textbf{Feature} & \textbf{Old Paradigm (Tyranny of Chronos)} & \textbf{New Pact (Symphony of Agents)} \\
\midrule
\textbf{Control}    & Centralized (teacher-led)                 & Decentralized (student-agent-led) \\
\textbf{Pacing}     & Uniform (cohort-based)                    & Personalized (mastery-based) \\
\textbf{Assessment} & Summative (grades for completed work)     & Formative and continuous (verifiable credentials) \\
\textbf{Feedback}   & Delayed and asynchronous                  & Immediate and contextual \\
\textbf{Teacher's Role} & Content deliverer / Task assigner      & Learning orchestrator / Mentor \\
\textbf{Student's Role} & Passive recipient / Task completer     & Active navigator / Problem solver \\
\textbf{Well-being} & High stress / Negative health impact      & Balanced / Monitored for well-being \\
\textbf{Equity}     & Reinforces existing inequalities          & Mitigates resource gaps \\
\bottomrule
\end{tabularx}
\end{table}

\section*{Preamble: A New Covenant for Learning: From Compulsion to Concert}
\addcontentsline{toc}{section}{Preamble: A New Covenant for Learning: From Compulsion to Concert}
This document is not merely a technical proposal; it is a response to a civilizational need. It stands as a manifesto for a new era in education, one in which technology serves to liberate human potential, not to mechanically format it. The contemporary education system, in its essence, resembles a cacophony---a collection of uncoordinated, often contradictory demands that clash in a student's life, creating dissonance and stress. At the heart of this cacophony is the homework paradigm, which relentlessly infringes upon the temporal sovereignty of young people, transforming their lives into an endless cycle of obligations.

In response to this crisis, we present a vision of transformation based on two pillars: the \enquote{Scholastic Pact on Temporal Non-Aggression} and the \enquote{Symphony of Agents} architecture. The Pact is our ethical commitment---a fundamental social contract that recognizes a student's time as a precious, protected resource intended for integral development, not for the thoughtless completion of tasks. The Symphony, in turn, is the technological realization of this Pact. It is a decentralized, intelligent ecosystem in which autonomous software agents, acting on behalf of students, teachers, and curricula, harmoniously orchestrate the educational process.

The central thesis of this document is as follows: by replacing the blunt instrument of mass-assigned homework with a decentralized, agent-driven system of personalized, just-in-time educational interventions, we can resolve the paradox of deteriorating student well-being in an age of unprecedented technological possibilities. This is not about abolishing effort, but about restoring its meaning. It is not about rejecting rigor, but about redefining it---as the pursuit of verifiable mastery, not as a measure of time spent on a task. We propose a shift from coercion to concert, from cacophony to symphony, in which every student, supported by their digital agent, can find their own unique melody of development. This is a blueprint for building an architecture for a more humane, just, and effective future of learning.

\section[The Tyranny of Chronos: Deconstructing the Homework Paradigm]{The Tyranny of Chronos\\Deconstructing the Homework Paradigm}
Before presenting the architecture of the solution, it is necessary to thoroughly understand the problem we intend to solve. This section builds an evidence-based argument that the current homework paradigm is not only ineffective but actively harmful, making radical change a moral and pedagogical imperative.

\subsection{The Quantified Student: Stress, Health, and the Erosion of Well-being}
The human cost of the current system is alarming and well-documented. Groundbreaking research by Denise Pope at Stanford University provides evidence of a negative correlation between excessive homework and the health and well-being of students, particularly in high-achieving academic environments \autocite{pope2013,pope_health}.

An analysis of survey data from 4{,}317 students at 10 renowned high schools in affluent California communities revealed a harsh reality \autocite{pope2013}. Students in these schools spend an average of 3.1 hours on homework each night \autocite{pope2013}. More alarmingly, 56\% of them identified homework as a primary source of stress---surpassing other stressors such as tests (43\%) or pressure to get good grades (33\%) \autocite{pope2013}. Less than 1\% stated that homework was not a source of stress at all \autocite{pope2013}.

The consequences of chronic stress extend beyond the psychological realm. In open-ended responses, students repeatedly reported that the burden of homework leads to serious health problems: chronic sleep deprivation, exhaustion, headaches, stomach problems, and even weight loss \autocite{pope2013,pope_health}. These physical symptoms are a direct result of a system that regularly forces young people to sacrifice sleep and rest for school obligations.

Equally severe are the social and developmental costs. Data indicate that an excess of homework prevents students from meeting developmental needs and cultivating other critical life skills \autocite{pope2013}. Students give up extracurricular activities, limit time with family and friends, and abandon hobbies that bring them joy. An education system that should support holistic development, in practice, inhibits it.

Moreover, this model undermines the very purpose of education. Research has shown no correlation between the amount of time spent on homework and the enjoyment derived from it \autocite{pope2013}. Students openly admit they often complete tasks they perceive as \enquote{pointless} or \enquote{mindless} solely to maintain grades \autocite{pope2013}. Such \enquote{busy work} discourages learning and instead promotes doing homework merely to earn points \autocite{pope2013}. In extreme cases, stress leads to unhealthy coping mechanisms such as cheating, substance abuse, or eating disorders \autocite{pope2022}.

\subsection{The Fallacy of Volume: Diminishing Returns and Entrenched Inequality}
The pedagogical argument for a large volume of homework also crumbles in the face of data. The belief that \enquote{more is better} proves to be a fallacy that not only fails to deliver expected benefits but also contributes to deepening social inequalities.

Prior work indicates that the benefits of homework plateau after roughly two hours per day, with the optimal time for high school students between 90 and 150 minutes \autocite{pope2013}. Each additional hour brings diminishing returns and increasing negative effects.

\subsection{A Call for a Temporal Non-Aggression Pact: Redefining \enquote{Effort} Beyond Time-on-Task}
This conclusion is supported by macro-level data. Analyses by the OECD show no clear link between average national homework hours and performance on PISA \autocite{borgonovi2014,birch2018}. Finland, a consistent educational leader, achieves excellent results with students spending about three hours per week on homework \autocite{borgonovi2014}. Conversely, high-performing countries like Singapore assign substantial homework, demonstrating the lack of a universal rule \autocite{borgonovi2014}.

A more detailed OECD analysis reveals a troubling pattern: within countries, students from wealthier backgrounds who do more homework tend to achieve higher scores \autocite{borgonovi2014}. However, this likely reflects that systems rely on homework to compensate for what could be learned in school, thereby advantaging those with resources (quiet study space, materials, educated parents). Homework thus becomes a mechanism for socio-economic sorting.

Data from Shanghai show students spending up to 14 hours per week on homework, yet additional hours among wealthier students (from 11 to 16) do not translate into further gains \autocite{borgonovi2014}, suggesting a ceiling of effectiveness. Subject and context matter: one U.S. study found significant benefits from math homework for eighth-graders, but not for English, science, or history; even in math, effects were stronger for white than Black students \autocite{borgonovi2014}.

Therefore, we formally propose establishing a \enquote{Scholastic Pact on Temporal Non-Aggression.} This Pact recognizes, inter alia, that the total learning time for a 15-year-old in Europe averages 43 hours per week (26 in school plus 17 on homework and self-study)---exceeding a full-time job---for questionable benefit \autocite{birch2018}. Claims that homework builds responsibility are undermined by research showing it often becomes a source of family stress and reduced shared leisure \autocite{birch2018}.

Moreover, the nature of most assigned homework is inconsistent with modern educational goals. Instead of creative application, complex problem solving, or teamwork, tasks tend to emphasize mechanical reinforcement of class material \autocite{birch2018}. This is precisely the kind of \enquote{mindless} work that leads to a disconnect between effort and mastery \autocite{pope2013}.

% TODO: If you want explicit Pact principles as a list, uncomment and edit.
% \begin{description}[leftmargin=2em,style=sameline]
%   \item[Respect for time:] Define protected windows for rest, family, and hobbies.
%   \item[Mastery-first:] Replace time-on-task metrics with verifiable mastery.
%   \item[Personalization:] Tailor interventions to needs and context, not cohorts.
%   \item[Well-being:] Continuously monitor and act on well-being indicators.
%   \item[Equity:] Design away resource-dependence and hidden homework subsidies.
% \end{description}

\subsection{The Voice of the Agent: Standardized Communication and Negotiation Protocols}
For heterogeneous agents (students, teachers, AI resources, curricula) to communicate effectively, a common Agent Communication Language is essential \autocite{dua2025_acl,labrou1999,fipa2002,sarl_github}. We propose adopting \gls{fipaacl}, which defines message structure, including:
- Performative (e.g., inform, request, \gls{cnp} cfp),
- Sender/Receiver,
- Content,
- Content descriptors (language, ontology, encoding),
- Conversation parameters (protocol, conversation-id).

Within this language, agents use interaction protocols. A fundamental one for task allocation is the \gls{cnp} \autocite{wiki_cnp,ohare1998}:
- Call for Proposals (CFP): the initiator broadcasts a need.
- Proposal Submission: participants respond with terms (e.g., time, cost).
- Evaluation and Selection: initiator chooses the best proposal.
- Acceptance/Rejection: winners get accept-proposal; others get reject-proposal.

Although the Contract Net protocol is a robust and proven mechanism, its classic form has limitations in dynamic, large-scale settings (communication overhead for frequent, low-cost interactions) \autocite{ohare1998}. This observation motivates a more lightweight orchestration mechanism.

\subsection{The Symphony Protocol: Dynamic Allocation of Educational Interventions via Beacons}
Our orchestration is inspired by the \enquote{Symphony} platform---a decentralized multi-agent system for scalable, privacy-preserving collaboration \autocite{wang2025,moonlight_symphony,emergent_mind,wang2025_2}. The key element is a beacon-selection protocol, a lightweight alternative to Contract Net \autocite{emergent_mind}.

Process \autocite{wang2025,moonlight_symphony}:
- Beacon Emission: an agent with a subtask (e.g., Student Agent needing a resource) broadcasts a short \enquote{Beacon} describing requirements (topic, level, style).
- Local Match Evaluation: available agents compute a local match score (0--1) against their capabilities.
- Best Match Selection: scores are returned; the initiator selects the best-matching agent to perform the task \autocite{moonlight_symphony}.

Compared to Contract Net, beacon-selection is more efficient for frequent, minor interactions, ideal for dynamically matching needs to resources in real time \autocite{emergent_mind}.

A pragmatic hybrid model uses:
- Beacon-selection for high-frequency, low-risk, automated matching (e.g., \enquote{find a 5-minute intro video on photosynthesis}).
- Contract Net for high-stakes, complex tasks (e.g., booking a 30-minute session with a human mentor).

\section{The Pact in Practice: A Strategic Transformation Plan}
Transforming the architecture into practice requires defining roles, illustrating workflows, and proposing a pragmatic, phased implementation strategy based on a \gls{mve} \autocite{adner2023,adner2023_2,msg_advisors,audi_weihe}.

\subsection{The Agent Ecosystem: Defining Roles in the New Learning Paradigm}
The Symphony depends on collaboration of specialized agents, each with a unique role---combining \gls{bdi} and \gls{fipaacl} with concrete educational functions \autocite{georgeff1998,wiki_bdi,rao_bdi,zhao2025}.

\textbf{\gls{student-agent}:} The student's autonomous assistant. Desire: achieve mastery within a defined curriculum. Beliefs: current mastery state (ledger-recorded). Intentions: active learning plans (use a resource, attempt a mastery challenge). It initiates interactions, negotiates access to resources, and manages the educational path.

\textbf{\gls{curriculum-agent}:} Digital representation of a knowledge domain (e.g., algebra, modern history). Stores the curriculum as a dependency graph, serves resources, and generates mastery challenges while preserving integrity and coherence.

\textbf{\gls{mentor-agent}:} Provider of explanations and guidance. Announces competencies in a registry. May be an AI agent or a proxy for a human teacher (managing calendar, specialization, responding to Beacons/CFPs). This empowers teachers to focus on high-value mentoring.

\textbf{\gls{wellbeing-agent}:} Monitors and supports well-being in line with the Pact. Desire: maintain indicators within healthy limits (learning time, frustration, breaks). Analyzes interaction metadata and can suggest breaks, exercises, or escalate to a human mentor.

\subsection{The Student Agent: Navigator of a Personalized Educational Path}
The Student Agent reframes the student as an active navigator of a personalized path toward verifiable mastery rather than time-on-task.

\subsubsection{Cognitive Architecture: BDI Logic in Practice}
Beliefs are based on ongoing assessment of mastery across topics. Intentions focus attention and resources on the next-best action (e.g., \enquote{practice factoring trinomials}).

\subsection{The Curriculum Agent: From Static Syllabi to Dynamic Graphs}
Unlike static curricula, the Curriculum Agent treats content as a living dependency graph of concepts, skills, and applications. It:
- monitors needs and the effectiveness of resources,
- prefers authentic mastery verification (application in new contexts),
- generates personalized challenges matched to level, style, history, and goals.

It communicates using \textbf{inform} (updates), \textbf{propose} (opportunities), and \textbf{confirm} (validations). The query-response protocol drives its operation: analyze context, retrieve/evaluate resources, and return curated responses with next-step suggestions.

\subsection{The Mentor Agent (AI): A Digital Socrates for Deep Understanding}
Goal: provide immediate, contextual guidance on demand. Beliefs are updated by analyzing learner difficulties to identify root causes and adapt explanations.

\paragraph{Illustrative flow.}
Failure and adaptation: A learner fails a challenge on factoring trinomials. The Student Agent updates its Beliefs and switches to a Contract Net CFP: \enquote{Seeking a 10-minute synchronous explanation on factoring trinomials; availability 4:00–6:00 PM.} A teacher’s proxy accepts; after a focused session, the learner retries and passes. The Curriculum Agent co-signs a transaction with the teacher’s proxy; a verifiable \enquote{Mastery: Quadratic Equations} credential is added to the learner’s ledger.

\subsection{Launching the Transformation: The Minimum Viable Ecosystem (MVE) Approach}
An \gls{mve} is the smallest configuration of interdependent actors that can deliver new value and grow sustainably \autocite{adner2023,adner2023_2,msg_advisors,audi_weihe}.

\paragraph{Core MVE.}
- Scope: one school, one subject (e.g., 9th-grade algebra).
- Actors: Student Agent (customer), Curriculum Agent (content/assessment), Mentor Agent (teacher proxies).

\paragraph{Value proposition.}
- Students: personalized, mastery-first, lower-stress paths.
- Teachers: less grading/assigning; more high-value mentoring.
- Administrators: real-time mastery data for informed decisions.

\paragraph{Scaling.}
- Horizontal: add subjects, then schools in a federated network.
- Vertical: introduce specialized agents (peer mentors, career pathways, group projects).
- Over time: toward a lifelong, portable educational network with verifiable credentials.

% ---------------------------
% Glossaries printout
\section{Glossary and Acronyms}
% Include all defined entries even if not referenced with \gls commands:
\glsaddall
\printglossary[title={Glossary},toctitle={Glossary}]
\printglossary[type=\acronymtype,title={Acronyms},toctitle={Acronyms}]

% ---------------------------
% References (biblatex)
\printbibliography[heading=bibintoc,title={References}]

\end{document}
